\subsectionaddtolist{آ}{

پردازه‌های A و B
به عنوان کلاینت عمل می‌کنند و از 
پورت‌های موقت استفاده می‌کنند. 
(Ephemeral-Port)
اما پردازه C که در این سناریو به عنوان سرور HTTP است، که باید با یک پورت ثابت کار کند. چون صورت سوال اشاره کرده است که در پروتوکل ارتباطی ما HTTP است، پس پورت پردازه C برابر با ۸۰ است.
}

\subsectionaddtolist{ب}{
	در حالت عادی، این امکان وجود ندارد. 
    چون که پورت و آیپی با هم 
    Socket-Address
    را تشکیل می‌دهند و باید یک کلید یکتا باشد. پس اگر یک پردازه بخواهد روی یک پورت خاص 
    bind کند،
    سیستم‌عامل آن پورت را برای این پردازه رزرو می‌کند و دیگر اجازه نمی‌دهد که پردازه دیگری روی آن پورت 
    bind
    کند. اما حالات استثنا و خاص هم وجود دارد، مانند:

    \begin{itemize}
        \item فرض کنیم یک هاست دارای چند آیپی است. می‌توان یک پردازه روی یک آیپی و پورت و پردازه دیگر روی همان پورت و آیپی دیگر گوش کند.
        \item با استفاده از قابلیت 
        SO-REUSEPORT
        می‌توان کاری کرد که دو پردازه روی یک پورت گوش دهند و سیستم‌عامل بین آن‌ها 
        load-balancing
        انجام دهد.
        \item و موارد دیگر...
    \end{itemize}
	
}

\subsectionaddtolist{ج}{
	
با اینکه در لایه‌ی لینک روش‌هایی برای تشخیص خطا وجود دارد،
اما این روش‌ها فقط خطا‌های local
را در هر لینک مجزا بین دو گره تشخیص می‌دهند.
در مسیر یک بسته از مبدا تا مقصد، بسته از چندین لینک و روتر عبور می‌کند و خطاهایی ممکن است در بافر روترها یا حافظه‌ی آن‌ها رخ دهد.
پس برای اطمینان از اینکه بسته به برنامه‌ی مقصد به درستی رسیده است، نیاز است در لایه‌ی انتقال یک checksum داشته باشیم.

	
}

