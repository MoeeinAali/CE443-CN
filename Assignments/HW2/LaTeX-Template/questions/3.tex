\subsectionaddtolist{آ}{

در پروتکل \textbf{Go-Back-N}، اندازه پنجره نقش بسیار مهمی در کارایی و کارکرد صحیح شبکه دارد. در صورتی که اندازه پنجره به درستی انتخاب نشود، دو حالت مشکل‌زا ممکن است رخ دهد:

\subsubsection*{پنجره بسیار کوچک}
\begin{itemize}
    \item کارایی پایین شبکه: ارسال‌کننده نمی‌تواند بسته‌های زیادی را قبل از دریافت تایید ارسال کند، که منجر به زمان‌های بیکار زیاد می‌شود.
    \item استفاده ناکارآمد از پهنای باند: ظرفیت لینک به خوبی استفاده نمی‌شود، زیرا اغلب اوقات ارسال‌کننده منتظر دریافت تأییدها است.
\end{itemize}

\subsubsection*{پنجره بسیار بزرگ}
\begin{itemize}
    \item \textbf{سرریز بافر گیرنده (Receiver-Buffer-Overflow)}: اگر گیرنده نتواند با سرعت کافی بسته‌ها را دریافت و پردازش کند، بسته‌ها از بین خواهند رفت.
    \item \textbf{افزایش احتمال ارسال مجدد بسته‌ها}: در صورت از دست رفتن یک بسته، همه بسته‌های بعدی باید دوباره ارسال شوند که باعث افزایش بار شبکه می‌شود.
\end{itemize}

اندازه بهینه پنجره معمولاً بر اساس رابطه زیر تعیین می‌شود:

\[
\text{Window-Size}_{\text{optimal}} = \text{Bandwidth-Delay-Product} + 1
\]

که در آن 
Bandwidth-Delay-Product بیانگر مقدار داده‌ای است که می‌تواند در مسیر شبکه در جریان باشد (در حال انتقال).
 عدد ۱ اضافه می‌شود تا اطمینان حاصل شود که حتی در هنگام انتظار برای تأیید آخرین بسته، ارسال‌کننده بتواند ارسال را ادامه دهد.

 انتخاب اندازه پنجره باید به گونه‌ای باشد که هم کارایی شبکه را افزایش دهد و هم از سرریز بافر جلوگیری کند. این اندازه معمولاً با محاسبه دقیق 
 BDP
  و اعمال حاشیه امن به دست می‌آید.
}


\subsectionaddtolist{ب}{
	
داده‌های مسئله:
\begin{itemize}
    \item RTT = \text{ms100}
    \item اندازه هر بسته = 1000 \text{بایت} = 8000 \text{بیت}
    \item نرخ لینک =  \text{Gbps} 1 = \text{bps} $10^9$ 
    \item احتمال خطا در هر بسته = $p$
\end{itemize}

زمان ارسال هر بسته:

\[
t_{\text{trans}} = \frac{8000}{10^9} = 8\ \mu s
\]

\subsubsection*{پروتکل Stop-and-Wait}
در Stop-and-Wait ،
 پس از ارسال هر بسته باید یک RTT صبر کنیم.

کارایی:

\[
U_{\text{SW}} = \frac{t_{\text{trans}}}{t_{\text{trans}} + RTT}
= \frac{8 \times 10^{-6}}{8 \times 10^{-6} + 0.1} \approx 7.999 \times 10^{-5}
\]

یعنی کارایی حدودا $0.008\%$ است.


کارایی با احتمال خطا $p$:

\[
U_{\text{SW}} = \frac{t_{\text{trans}}}{t_{\text{trans}} + RTT} \times (1 - p)
\]

\[
U_{\text{SW}} \approx 7.999 \times 10^{-5} \times (1 - p)
\]


\subsubsection*{پروتکل Go-Back-N}
ابتدا اندازه پنجره $N$ را محاسبه می‌کنیم:

\[
N = \frac{RTT}{t_{\text{trans}}} = \frac{0.1}{8 \times 10^{-6}} = 12500
\]

بنابراین، $N=12500$ برای پر کردن خط لوله کافی است.

در حالت بدون خطا:

\[
U_{\text{GBN}} = \frac{N \times t_{\text{trans}}}{RTT + t_{\text{trans}}} \approx \frac{12500 \times 8 \times 10^{-6}}{0.1} = 1
\]

یعنی کارایی ۱۰۰\% در حالت بدون خطا.

در حضور خطا با احتمال $p$، کارایی کاهش می‌یابد. کارایی تقریبی:

\[
U_{\text{GBN}} \approx \frac{1}{1 + p(N-1)}
\]

\subsubsection*{Selective Repeat}
در این حالت نیز پنجره $N$ به همان مقدار است.

در حالت بدون خطا:

\[
U_{\text{SR}} = 1
\]

در حضور خطا:

\[
U_{\text{SR}} \approx 1 - p
\]

زیرا فقط بسته‌های دارای خطا دوباره ارسال می‌شوند.

\subsubsection*{با افزایش p :}
\begin{itemize}
    \item در {Stop-and-Wait} ، کارایی به شدت پایین است و تغییر زیادی با خطا نمی‌کند زیرا در بهترین حالت هم کارایی بسیار کم است.
    \item در {Go-Back-N} ، با افزایش $p$، کارایی شدیدا افت می‌کند زیرا باید همه بسته‌های بعد از یک خطا را دوباره ارسال کرد.
    \item در {Selective-Repeat} ، افت کارایی بسیار ملایم‌تر است، زیرا فقط بسته‌های معیوب بازفرست می‌شوند.
\end{itemize}

برای لینک‌هایی که احتمال خطا در آن‌ها زیاد است، استفاده از پروتکل {Selective-Repeat} مناسب‌تر است زیرا کارایی آن در مقابل خطا مقاوم‌تر است.


}

\subsectionaddtolist{ج}{

\subsubsection*{پروتکل Go-Back-N}

در این پروتکل اگر یک بسته گم شود، تمام بسته‌های بعدی (تا انتهای پنجره) باید دوباره ارسال شوند.

تعداد متوسط بسته‌های ارسالی:

\[
\text{Expected-transmissions-per-packet} = \frac{1}{1 - p}
\]

اما، چون در Go-Back-N از لحظه‌ای که یک خطا رخ دهد تمام بسته‌های بعدی نیز دوباره ارسال می‌شوند، میانگین تعداد بسته‌های اضافی در اثر خطا تقریباً:

\[
\text{Overhead-per-lost-packet} \approx N - 1
\]

پس، تعداد متوسط بسته‌های ارسالی برای ارسال $M$ بسته:

\[
\text{Total-packets} \approx M \times \left( \frac{1}{1 - p} + p \times (N - 1) \right)
\]

تعداد متوسط ACK ها:

\[
\text{Total-ACKs} = M
\]

چون فقط بسته‌های دریافت‌شده نیازمند ACK هستند و ACK ها گم نمی‌شوند.

\pagebreak
\subsubsection*{پروتکل Selective Repeat}

\[
\text{Expected-transmissions-per-packet} = \frac{1}{1 - p}
\]

پس، برای $M$ بسته:

\[
\text{Total-packets} = M \times \frac{1}{1 - p}
\]

تعداد متوسط ACK ها:

\[
\text{Total-ACKs} = M
\]

چون هر بسته‌ای که دریافت می‌شود، بلافاصله ACK می‌شود.

\subsubsection*{پروتکل TCP}

در TCP بدون Delayed ACK برای هر بسته دریافت‌شده، بلافاصله یک ACK ارسال می‌شود.

تعداد متوسط بسته‌های ارسالی:

TCP از Selective Repeat مشابه‌تر است. در صورت از دست رفتن بسته، تنها همان بسته بازفرست می‌شود:

\[
\text{Expected-transmissions-per-packet} = \frac{1}{1 - p}
\]

پس:

\[
\text{Total-packets} = M \times \frac{1}{1 - p}
\]

تعداد متوسط ACK ها:

\[
\text{Total-ACKs} = M
\]

\subsubsection*{نتیجه‌گیری}

\begin{itemize}
    \item در Go-Back-N ، به علت ارسال مجدد کل پنجره پس از هر خطا، تعداد بسته‌های ارسالی بسیار بیشتر از Selective-Repeat و TCP است.
    \item در Selective-Repeat و TCP ، فقط بسته‌های از دست‌رفته دوباره ارسال می‌شوند و تعداد ACK ها برای هر سه پروتکل برابر با تعداد کل بسته‌ها است.
    \item بنابراین Selective-Repeat و TCP بهینه‌ترین عملکرد را از نظر تعداد بسته‌های ارسالی دارند.
\end{itemize}
	
}

