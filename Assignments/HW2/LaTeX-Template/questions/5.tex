\subsectionaddtolist{آ}{

این دستور قصد دارد آدرس آیپی متناظر با دامنه \lr{sharif.edu} را پیدا کند.
با این تفاوت که از یک DNS سرور با آدرس 
\lr{4.2.2.4}
استفاده می‌کند نه از DNS سرور پیش‌فرض سیستم.

پاسخ داده شده دارای اطلاعات زیر هست:

\begin{itemize}
    \item آدرس آیپی ورژن ۴ متناظر با این دامنه
    \item مقدار TTL مربوط به این آدرس آیپی
    . این مقدار برابر ۸ است در این درخواست.
    \item زمان پاسخ سرور DNS برابر با 221 msec است.
    \item پاسخ روی پورت 53 و پروتوکل UDP ارسال شده است.
    \item زمان اجرای دستور. البته احتمالا در این مورد زمان ست نشده بوده و در جواب زمان Unix-Timestamp برگردانده شده است.
    \item اندازه‌ی پیام ارسال شده از سمت سرور که در این جواب 55 بایت است.
    \item عبارت Truncated یعنی پاسخ DNS خیلی بزرگ بوده، داخل UDP جا نشده، و dig فقط بخشی از آن را نشان داده است.
\end{itemize}

}

\subsectionaddtolist{ب}{
	رکورد MX مشخص می‌کند که ایمیل‌هایی که به این دامنه ارسال می‌شوند، باید به کدام سرورهای ایمیل تحویل داده شوند.
    این درخواست از DNS سرور نوشته شده درخواست می‌کند تا رکوردهای MX دامنه‌ی
    \lr{sharif.edu}
    را پیدا کند.

    پاسخ داده شده دارای اطلاعات زیر هست:
\begin{itemize}
    \item این دامنه دارای دو رکورد MX است. 
    \item این رکوردها هردو دارای اولویت 10 هستند.
    \item این رکوردها دارای TTL برابر 60 هستند.
    \item باقی اطلاعات مشابه با بخش قبل هستند.
\end{itemize}

وقتی ایمیلی ارسال می‌شود، سرور فرستنده تلاش می‌کند ایمیل را به سروری با کمترین عدد اولویت بفرستد. اگر چند سرور اولویت یکسان داشته باشند، یکی از آن‌ها انتخاب می‌شود (اغلب به طور تصادفی یا بر اساس load-balancing ).
	
}


