\subsectionaddtolist{آ}

کل بسته دارای 3000 بایت است و هدر ما 20 بایت، پس دیتا 2980 بایت است. از طرفی می‌دانیم که هر فرگمنت 1480 بایت داده منتقل می‌کند. پس حداقل به 3 فرگمنت برای این کار نیاز داریم.

\[
3 > \frac{2980}{1480} > 2
\]

\subsectionaddtolist{ب}


از بین روترهای A و B و C روتر C بیشترین پیشوند مشترک را دارد و بسته به آن روتر فرستاده می‌شود. همچنین داخل شبکه روتر D قرار ندارد و اصلا مچ نمیشود.

\subsectionaddtolist{ج}

\begin{itemize}
	\item \textbf{درست.}
	پروتوکل $IPv6$ با داشتن فضای آدرس‌دهی بسیار بزرگ، نیاز به ترجمه آدرس شبکه که در $IPv4$ برای جبران کمبود آدرس‌ها استفاده می‌شود را کاهش می‌دهد.
	\item \textbf{درست.}
	تونلینگ به عنوان روشی برای ارسال بسته‌های یک پروتکل داخل پروتکل دیگر استفاده می‌شود که در VPN ها و همچنین برای عبور $IPv6$ روی زیرساخت $IPv4$ کاربرد دارد.
\end{itemize}



\subsectionaddtolist{د}


\begin{itemize}
	\item \textbf{ضعف.}
چون NAT باعث می‌شود دستگاه‌ها در داخل شبکه محلی آدرس‌های خصوصی داشته باشند و همه آنها پشت یک آدرس عمومی مشترک مخفی شوند، این موضوع مدیریت و شناسایی دستگاه‌ها را پیچیده‌تر می‌کند.
	\item \textbf{ضعف.}
زیرا NAT نیاز به پردازش اضافه برای ترجمه آدرس‌ها دارد که ممکن است تاخیر کمی ایجاد کند و همچنین پنهان کردن آدرس‌های واقعی باعث می‌شود عیب‌یابی شبکه دشوارتر شود.

\end{itemize}

