\subsectionaddtolist{الف}{


پایین آوردن سریع بیت‌ریت در سطح بافر بحرانی به موقع است. با پایین آوردن بیت‌ریت به 1 Mbps ، احتمال پر شدن بافر و جلوگیری از توقف پخش افزایش می‌یابد. استفاده از اطلاعات بیت‌ریت قبلی و پهنای باند جاری کمک می‌کند که بیت‌ریت انتخابی با شرایط واقعی شبکه تطبیق داشته باشد. الگوریتم فقط به وضعیت فعلی نگاه می‌کند (نه تغییرات آتی یا روند پهنای باند). این باعث می‌شود در مواجهه با نوسانات سریع، پیش‌بینی‌پذیری پایین بیاید.
همچنین افزایش یا کاهش‌های بیش از حد کیفیت در این الگوریتم به همراه داریم.
الگوریتم در بیشتر شرایط پایدار یا نیمه‌پایدار پهنای باند عملکرد خوبی دارد و از Rebuffering جلوگیری می‌کند.
اما اگر پهنای باند به طور ناگهانی و سریع کاهش یابد و سطح بافر هنوز بالای 5 ثانیه باشد، ممکن است الگوریتم به اندازه کافی سریع واکنش نشان ندهد، و خطر Rebuffering به‌وجود بیاید.

}



\subsectionaddtolist{ب}{
	به عنوان مثال:

		پهنای باند در دسترس: $B_t = 6 Mbps$
		
		 سطح بافر: $L_t = 6 s$
		 
		 بیت‌ریت آخرین قطعه: $R_t = 4 Mbps$
	
	چون $L_t > 5 s$ است، شرط اول برقرار می‌شود و بیت‌ریت به پایین‌ترین مقدار کاهش می‌یابد. از آنجا که ما اکنون با بیت‌ریت 1 شروع کرده‌ایم، حتی اگر بافر کم شود یا پهنای باند بالا باشد، شرط ارتقا هیچ‌گاه فعال نمی‌شود (چون $R_t/2 = 0.5 < B_t$) و در نتیجه سیستم در 1 Mbps قفل می‌ماند.
	
	معمولا وقتی بافر کم است، باید بیت‌ریت را کاهش دهیم و وقتی بافر بسیار بالاست، می‌توانیم به‌تدریج افزایش دهیم. اما اینجا برعکس پیاده شده، به محض رسیدن بافر بالاتر از ۵ ثانیه، بیت‌ریت ناگهان به حداقل افت می‌کند، حتی وقتی شبکه خیلی خوب است.
	
	 پیشنهاد بهبود:
	استفاده از میانگین‌گیری از وضعیت پهنای‌باند و بافر در چند لحظه به ‌جای بررسی فقط یک لحظه‌ی خاص.
	
}

\subsectionaddtolist{ج}{
	
به عنوان مثال:
	
	پهنای باند ثابت: $B = 3.5 Mbps$
	
	طول هر قطعه ویدئو: 2 ثانیه
	
	شروع با بیت‌ریت $R = 2Mbps$ و بافر اولیه $L = 7s$
	
	
سطح بافر دائم بین خالی شدن تا زیر $5s$ و پر شدن تا بالای $10s$ نوسان می‌کند و کیفیت ویدئو پیوسته تغییر می‌کند.

پیشنهادات برای رفع این مشکل:

\begin{itemize}
	\item استفاده از Smoothing-Filter به این صورت که به جای استفاده از $B_t$ لحظه ای، از میانگین متحرک استفاده کند.
	\item یک شرط اضافه کنیم که یک حداکثر تغییراتی را بررسی کند تا ویدیو یک دفعه از بالاترین کیفیت به پایین ترین کیفیت نپرد.
\end{itemize}
}