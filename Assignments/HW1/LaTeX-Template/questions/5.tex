فرض میکنیم که ریکوئست‌ها به صورت نرمال در 2 دقیقه زده شده‌اند. پس بین هر 2 ریکوئست 12 ثانیه فاصله است.


\subsectionaddtolist{الف}{


پرسش به ریشه و دریافت ارجاع به TLD :
$50ms$

پرسش به سرور TLD و دریافت ارجاع به Authoritative :
$40ms$

پرسش نهایی برای آدرس www.example.com و دریافت پاسخ نهایی:
$30ms$

بازگشت پاسخ نهایی به کلاینت:
$10ms$


پس در مجموع $130ms$ برای دریافت نتیجه زمان لازم است.
}

\subsectionaddtolist{ب}{
	در این صورت درخواست اول و ششم کش ندارند و باید $130ms$ منتظر بمانیم. اما باقی دستورات در DNS محلی کش شده‌اند و هر کدام $10ms$ زمان نیاز دارند.
	پس:
	
	\setLTR
	$2\times130 + 8\times10 = 340ms$
	\setRTL
}

\subsectionaddtolist{ج}{
	در این صورت درخواست‌های اول و چهارم و هفتم و دهم کش ندارند و هر کدام $130ms$ زمان نیاز دارند. باقی درخواست‌ها هر کدام $10ms$ نیاز دارند. پس:
	
\setLTR
$4\times130 + 6\times10 = 580ms$
\setRTL

}

\subsectionaddtolist{د}{
	
	\subsubsection*{افزایش TTL در Authoritative-DNS}
	
	در این صورت مدت زمان بیشتر آدرس کش میشود و نیاز نیست به Root DNS ریکوئست زده شود. البته از معایب این کار هم دیرتر اپدیت شدن تغییر ip است. البته میتوان با یک عدد معقول حداکثر بهره را از کش کردن برد. 
	
	\subsubsection*{استفاده از کش سمت مرورگر کاربر}
	در حال حاضر فقط سرور DNS محلی کش دارد (به مدت 60 ثانیه). اگر کلاینت یا مرورگر نیز بتواند نتایج resolve شده را در سطح خودش کش کند، دیگر نیازی به تماس مجدد با سرور DNS محلی در هر درخواست نخواهد بود.
	
	\subsubsection*{استفاده از DNS-Prefetching در مرورگر}
	مرورگر یا برنامه، دامنه‌هایی که احتمال میده کاربر به اون‌ها مراجعه کنه رو از قبل resolve می‌کنه.
	
}









