زمان مورد نیاز برای جست‌وجوی DNS به این صورت است:

\setLTR
$
RTT_1 + RTT_2 + RTT_3 + RTT_4 = 250 + 150 + 100 + 50 = 500ms
$
\setRTL


\subsectionaddtolist{الف}{

در این حالت به ازای هر فایل رو کانکشن TCP تشکیل داده و فایل را دریافت میکنیم. به ازای هر فایل هم به 2 RTT نیاز داریم. یکی برای تشکیل کانکشن TCP و یکی هم برای دریافت فایل. پس در مجموع:

\setLTR
$
DNS + 3 \times (2 \times RTT_0) = 500 + 6 \times 300 = 2300ms
$
\setRTL

}

\subsectionaddtolist{ب}{
	
	در این حالت هر سه فایل موازی و با یک کانکشن مخصوص به خود دریافت میشوند، پس برای هرسه فایل نیاز به 2 RTT داریم. زمان DNS هم که همچنان نیاز است، پس:
	
	\setLTR
	$
	DNS + (RTT_0 + RTT_0) = 300 + 300 + 500 = 1100ms
	$
	\setRTL
	
	
}

\subsectionaddtolist{ج}{
	در این حالت فقط 1 کانکشن TCP داریم که همه فایل ها یکی یکی از همین دانلود میشوند.  پس یک DNS داریم و یک RTT برای اتصال TCP و به ازای هر فایل هم یک  RTT . پس:
	
	
		\setLTR
	$
	DNS + (RTT_0 + RTT_0 + RTT_0 + RTT_0) = 500 + 1200 = 1700ms
	$
	\setRTL
}



