\subsectionaddtolist{الف}{
	
\subsubsection*{1.}
میزان تاخیر از مبدا تا اولین سوییچ:

\setLTR
$delay = \frac{L}{R} = \frac{10^6}{5\times10^6} = 0.2s$
\setRTL

\subsubsection*{2.}

هر سوییچ بسته را به طور کامل دریافت و سپس ارسال میکند، پس تا مقصد نهایی باید 3 مرتبه تاخیر مرتبه قبل را حساب کنیم:

\setLTR
$delay = 3\times \frac{L}{R} = \frac{10^6}{5\times10^6} = 0.6s$
\setRTL

\subsubsection*{3.}

اگر N ایستگاه در مسیر داشته باشیم، پس N + 1 مسیر داریم، پس در نهایت هم N + 1 تاخیر داریم:

\setLTR
$delay = (N+1)\times \frac{L}{R} = \frac{10^6}{5\times10^6} =(N+1)\times 0.2s$
\setRTL

}


\subsectionaddtolist{ب}{
	
	
	\subsubsection*{1.}
	میزان تاخیر اولین بسته به این صورت است:
	
	\setLTR
	$delay_{packet_1} = \frac{L}{R} = \frac{10^4}{5\times10^6} = 2ms$
	\setRTL

	
	\subsubsection*{2.}
	همزمان با دریافت کامل بسته اول، بسته دوم ارسال میشود که آن هم 2 میلی ثانیه بعد به اولین سوییچ میرسد. پس:
	
	\setLTR
	$delay_{packet_2} = 2 \times delay_{packet_1} = 4ms$
	\setRTL
	
		\subsubsection*{3.}
به تعمیم بخش قبل، زمان رسیدن بسته K ام به اولین سوییچ برابر است با:
	
	\setLTR
	$delay_{packet_K} = K \times delay_{packet_1} = 2K \ ms$
	\setRTL
	
}



\subsectionaddtolist{ج}{
\subsubsection*{1.}

	میدانیم پکت آخر در $t=200ms$ در اولین سوییچ است، پس $4ms$ بعد در زمان $t=204ms$ به مقصد میرسد. همچنین بدون شکستن پیام، نیاز به $600ms$ داشتیم.
	
این اتفاق به این دلیل است که بخشی از انتقال پیام ها به صورت موازی جلو میرود. هنگامی که پکت اول در حال انتقال از سوییچ دوم به سوم است، پکت دوم از سوییچ اول به دوم منتقل میشود و ...

واضحا هرچه تعداد سوییچ های در میان راه بیشتر باشد، این اختلاف زمانی بیشتر خواهد شد.

\subsubsection*{2.}

در صورتی که سایز بسته ها متفاوت باشد، زمان ارسال آن ها هم متفاوت خواهد بود و بر روی زمان کل تاخیر اورهد دارد. چون بسته های بزرگتر زمان بیشتری صرف ارسال میکنند و باعث کم شدن موازی سازی بسته های کوچک تر هم میشوند. نیاز به صف و تاخیر صف هم خواهیم داشت در این صورت. اما باز هم این حالت بهتر از حالت اول است که کل پیغام را یکجا ارسال کردیم!

}


\subsectionaddtolist{د}{
	\subsubsection*{1.}
	
	در این حالت یکی از مسیرها زمان بیشتری از $s0.2$ نیاز دارد، پس:
	
	\setLTR
	$
	delay = \frac{L}{R_1} + 2 \times \frac{L}{R_2} = 0.5 + 2 \times 0.2  = 0.9 s = 900 ms
	$
	\setRTL
	
	در حالت تقسیم بسته هم نیاز به $510ms$ زمان داریم. 
	
	\setLTR
	$
	delay = \frac{10000}{2\times10^6} = 0.005 = 5ms \rightarrow 100\times5 + 2\times5 = 510ms
	$
	\setRTL
	
		\subsubsection*{2.}
		
		اگر ﺑﺎﻓﺮ ما ﻣﺤﺪﻭﺩ ﺑﺎﺷﺪ آنوقت تعدادی از بسته ها دراپ میشوند و نیاز است دوباره ارسال شوند که باعث افزایش تاخیر میشود.
		
}


\subsectionaddtolist{ه}{
\subsubsection*{1.}
در حالت تقسیم پیام، به هر 100 بسته یک هدر 100 بیتی اضافه میشود پس کل زمان ارسال برابر است با:

	\setLTR
$
delay = (100 + 2) \times \frac{100 + 10000}{5\times10^6} = 206.04ms 
$
\setRTL

در حالت ارسال بدون تقسیم بندی، این سربار زمانی به مراتب کمتر و ناچیز خواهد بود:

	\setLTR
$
delay = 3 \times \frac{100 + 1000000}{5\times10^6} = 3\times 200.02ms = 600.06 ms
$
\setRTL

\subsubsection*{2.}

مکانیزم های زیادی وجود دارد، به عنوان مثال اگر یکی از پکت ها دچار مشکل شود و خراب ارسال شود، نیاز نیست کل پکت ها مجدد ارسال شوند و ما میتوانیم پکت خراب را شناسایی کنیم و فقط آن را دریافت کنیم. میتوان در هر پکت اطلاعاتی اضافی قرار بدیم تا در صورت نیاز خطای آن را اصلاح کرد! البته هردوی این روش ها مقداری سربار دارند و ما داریم اطلاعات اضافی در هر پکت قرار میدهیم، اما به طور کلی باعث بهبود اوضاع میشود و مفید است.


}


\subsectionaddtolist{و}{

\subsubsection*{1.}

اگر یکی از لینک‌های شبکه دارای پهنای باندی متغیر بین ۲ تا ۵ مگابیت بر ثانیه باشد، در حالتی که پیام به صورت یکپارچه و بدون تقسیم ارسال شود، ممکن است کل پیام در زمانی منتقل شود که لینک در کمترین پهنای باند خود قرار دارد. این موضوع باعث افزایش چشم‌گیر زمان انتقال می‌شود.

اما اگر پیام به بخش‌های کوچکتر تقسیم شود، این امکان به وجود می‌آید که بسته‌ها در زمان‌هایی که پهنای باند بالاتر است، منتقل شوند. در نتیجه، بخشی از داده‌ها سریع‌تر جابجا شده و زمان کلی انتقال کاهش می‌یابد.

\subsubsection*{2.}

در چنین شرایطی، استفاده از Adaptive-Segmentation و تنظیم پویا اندازه بسته‌ها بر اساس شرایط لحظه‌ای شبکه، می‌تواند بهره‌وری انتقال را به شکل قابل توجهی افزایش دهد و عملکرد کلی شبکه را بهبود بخشد. به این صورت که در زمان ازدحام، بسته های کوچک تری ارسال میشود.

}