پهنای باند اختصاص داده شده به هر اتصال:

\setLTR
$r = \frac{500}{N} bps$
\setRTL

زمان ارسال یک بسته کنترلی:

\setLTR
$t_{ctrl} = \frac{500bit}{\frac{500}{N} bps} = N s$
\setRTL

زمان ارسال یک بسته داده کوچک:

\setLTR
$t_{100k} = \frac{100000}{\frac{500}{N}} = 200N s $
\setRTL

زمان ارسال یک بسته داده بزرگ:

\setLTR
$t_{300k} = \frac{300000}{\frac{500}{N}} = 600N s $
\setRTL

\subsectionaddtolist{الف}{

زمان‌هایی که در این حالت باید بررسی کنیم عبارتند از: 

\begin{itemize}
	\item سه بسته کنترل برای TCP Handshaking
	\item یک بسته کنترل برای GET + تاخیر پردازش
	\item ارسال داده
	\item بسته کنترلی برای ACK نهایی
\end{itemize}

به هر یک از تاخیرهای بالا یک تاخیر صف هم اضافه خواهد شد.


\setLTR
$T = 3\times (t_{ctrl} + d_{queue}) + (t_{ctrl} + d_{queue} + d_{proc}) + (t_{data} + d_{queue}) + (t_{ctrl} + d_{queue}) = 5N + 0.1 + 6\times 0.05 + t_{data}$
\setRTL

با توجه به اینکه نسبت تعداد بسته های کوچک و بزرگ برابر است، پس بین طول آن ها میانگین گرفته و فرض میکنیم که بسته های ما 200000 بیت هستند. پس زمان ارسال آن ها برابر است با:
$t_{200k} = \frac{200000}{\frac{500}{N}} = 400N s $

پس در نهایت زمان ما برای یک شی برابر است با:

\setLTR
$ 5N + 0.1 + 6\times 0.05 + t_{data} = (405N + 0.4) s$
\setRTL


ما در مجموع 21 شی داریم، یکی فایل html و 20 فایل ارجاع شده. و چون N اتصال موازی هستند، پس زمان نهایی برابر است با:

\setLTR
$ T_{total} = \frac{21\times (405N + 0.4)}{N} = 8505 + \frac{8.4}{N}$
\setRTL


با توجه به سربار اتصال های موازی و ثابت بودن زمان 8505 ثانیه ای، افزایش N عملا کمکی به کاهش زمان کل دانلود نمی‌کند. پس این کار خیلی منطقی نیست.


}




\subsectionaddtolist{ب}{
	
	به صورت شهودی واضح است که بهبود قابل توجهی حاصل نخواهد شد. چون بخش بیشتر زمان دانلود به دلیل حجم بالای داده ها و لینک بسیار کم سرعت است. با HTTP پایا مقدار کمی بهبود سرعت داریم اما در برابر زمان انتقال داده ها بسیار ناچیز است. پس با این کار زمان دانلود مقدار کمی بهبود می یابد، اما آنقدر قابل توجه نیست. برای بهبود سرعت یا باید ظرفیت لینک را افزایش داد یا حجم داده ها را کم کرد. 
	
}


