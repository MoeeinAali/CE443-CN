میدانیم که:

\setLTR
$
D_{CS} \ge max\{\frac{NF}{u_s} , \frac{F}{d_{min}}\} \\
D_{P2P} \ge max\{\frac{F}{u_s},\frac{F}{d_{min}},\frac{NF}{u_s + \sum u_i}\}
$
\setRTL

همچنین:

\setLTR
$
F = 40Gbit = 40000Mbit \\ d = 2Mbps \\ u_s = 20Mbps
$
\setRTL


\subsectionaddtolist{الف}{


در حالت Client-Server داریم:

\setLTR
$D_{CS} \ge max\{\frac{N \times 40 \times 1024}{20} , \frac{40 \times 1024}{2}\} = max\{2048N,20480\}$

\setRTL

	{\centering{
		
		\begin{tabular}{|c|c|}
			\hline
			N    & Delay \\ \hline
			10   & 20480s + 50ms     \\ \hline
			100  & 204800s + 50ms   \\ \hline
			1000 & 2048000s + 50ms  \\ \hline
		\end{tabular}
		
}}


در حالت P2P فرض میکنیم اپلود همگن است، پس: 
$\sum u_i = N \times u$

پس برای این حالت از این رابطه استفاده میکنیم:

\setLTR
$
D_{P2P} \ge max\{ \frac{40960}{20} , \frac{40960}{2} , \frac{40960N}{20 + Nu}  \} = max\{20480 , 2048 , \frac{40960N}{20 + Nu}\}
$
\setRTL


	{\centering{

\begin{tabular}{|c|c|c|}
	\hline
	N    & u(Mbps)   & Delay(s) \\ \hline
	10   & 3.0 & 20   \\ \hline
	10   & 7.0 & 20   \\ \hline
	10   & 2   & 20   \\ \hline
	100  & 3.0 & 82   \\ \hline
	100  & 7.0 & 46   \\ \hline
	100  & 2   & 20   \\ \hline
	1000 & 3.0 & 128  \\ \hline
	1000 & 7.0 & 57   \\ \hline
	1000 & 2   & 20   \\ \hline
\end{tabular}
		
}}

پس در $N=10$ تاخیر به زمان دانلود محدود میشود، اما در N های بیشتر، حالت P2P نتیجه بهتری دارد.

}

\subsectionaddtolist{ب}{
	
	وقتی ۲۰٪ از همتاها هر ۵ دقیقه از شبکه جدا می‌شوند، بلوک‌هایی که دریافت کرده‌اند در شبکه باقی نمی‌مانند و مجبوریم آن‌ها را مجدداً از سرور یا دیگر همتاها درخواست کنیم. این امر باعث بار اضافه روی سرور و طولانی‌تر شدن توزیع می‌شود.
	
	اگر نرخ آپلود هر همتا از توزیع نرمال با $\sigma = 0.2 u$ پیروی کند، برخی پییرها کندتر و برخی سریع‌تر عمل می‌کنند. نبود تعادل، جریان گردش بلوک‌ها را مختل و زمان کل را به‌طور قابل توجهی افزایش می‌دهد.
	
	پهنای باند سرور با دوره‌ی ۱۰ دقیقه و دامنه‌ی ±۱۰٪ نوسان دارد. در فازهای افت، ظرفیت ارسال کمتر شده و تأمین بلوک‌های اولیه کند می‌شود که زمان توزیع را بیشتر می‌کند.
	
	RTT
	 حدود $50 ms$ بین سرور و همتا و $100ms$ بین همتاها باعث تأخیر در ارسال ACK و درخواست بلوک بعدی می‌شود. این انتظار اضافی نرخ گردش داده را کاهش و توزیع را کندتر می‌کند.
	
	با ترکیب این چهار عامل، توزیع فایل از چند دقیقه‌ی ایده‌آل به ده‌ها دقیقه در سناریوی واقعی کشیده می‌شود.
}

\subsectionaddtolist{ج}{
	
	\begin{itemize}
		\item ایجاد انگیزه برای ماندگاری: با مکانیزم اعتبار و پاداش‌دهی ، کاربران را تشویق کنید پس از دریافت کامل فایل همچنان فعال بمانند.
		\item 
		یک یا چند همتای قدرتمند با پهنای باند بالا را همیشه به‌عنوان seed نگه دارید تا هر زمان churn بالا رفت بتوانند جایگزین بلوک‌های از دست‌رفته شوند.
		\item هر همتا پیش از ترک شبکه چند بلوک اضافی برای همتاهای دیگر دانلود و نگهداری کند تا در دوره‌های خروج انبوه، قطع سرویس نداشته باشیم.
	
	\item همتاها را بر اساس تاخیر شبکه گروه‌بندی کنید و در اولویت با همتاهای نزدیک‌تر یا با RTT کمتر دانلود/آپلود کنید تا مدت زمان انتقال هر بلوک کوتاه‌تر شود.
	
	\end{itemize}
	
	
	
}