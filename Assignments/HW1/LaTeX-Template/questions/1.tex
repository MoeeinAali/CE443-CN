\subsectionaddtolist{الف}{



\begin{table}[ht]
	\centering
	\scriptsize
	\begin{tabular}{@{}lccccc@{}}
		\toprule
		\textbf{معیار / فناوری} & \textbf{DSL} & \textbf{کابل} & \textbf{FTTH} & \textbf{Wi-Fi} & \textbf{{G4 / G5}} \\ 
		\midrule
		\textbf{پهنای باند} & تا 100 Mbps & تا 1 Gbps & تا 10 Gbps+ & تا 1 Gbps & G4: Gbps 1 \\ 
		& & & & & G5: Gbps 10 \\
		\textbf{تأخیر} & ms 20–50  & ms 10–30 & ms 1–10 & ms 5–30 & G4 : ms 30–50 \\ 
		& & & & & G5 : ms 1–10 \\
		\textbf{مقیاس‌پذیری} & محدود & متوسط & بسیار بالا & محلی & بالا \\
		\textbf{هزینه} & کم & متوسط & بالا & کم & بالا \\
		\textbf{پوشش} & گسترده & شهری خوب & محدود به مناطق مجهز & محیط داخلی & بسیار گسترده \\
		\bottomrule
	\end{tabular}
	\caption{مقایسه‌ی فناوری‌های شبکه دسترسی از جنبه‌های مختلف}
\end{table}


}


\subsectionaddtolist{ب}{

	برای هر یک از این مناطق جغرایی یک پیشنهاد دارم که در ادامه شرح می‌دهم.
	
	\subsubsection*{1. مناطق شهری با تراکم بالا}
	\textbf{پیشنهاد: FTTH }
	
	دلایل انتخاب:
	\begin{itemize}
		\item کاربران پرمصرف (استریم، بازی، دورکاری) به سرعت بالا و تأخیر کم نیاز دارند.
		\item تراکم بالا، هزینه سرانه‌ی نصب فیبر را کاهش می‌دهد و توجیه‌پذیر می‌کند.
		\item مقیاس‌پذیری بالای FTTH امکان رشد آینده را فراهم می‌کند.
	\end{itemize}

	\subsubsection*{2.  مناطق روستایی}
	\textbf{پیشنهاد: کابل ترکیبی}
	
	دلایل انتخاب:
	\begin{itemize}
		\item هزینه‌ی پایین‌تر نسبت به FTTH در مناطقی با تراکم کمتر.
		\item امکان استفاده از زیرساخت‌های موجود (در   صورت وجود) مانند کابل تلویزیون و تلفن.
		\item عملکرد قابل‌قبول برای کاربردهایی مثل ویدئو کنفرانس و استریم با کیفیت متوسط تا بالا.
	\end{itemize}
	
	\subsubsection*{3. مناطق دورافتاده}
	\textbf{پیشنهاد: اتصال بی‌‌سیم G5 و G4}
	
	دلایل پیشنهاد:
	\begin{itemize}
		\item نصب فیبر یا کابل در این مناطق هزینه‌بر و غیرعملی است.
		\item دکل‌های سلولی می‌توانند مساحت وسیعی را پوشش دهند.
		\item با نصب تجهیزات قوی‌تر کاربران می‌توانند به خدمات با کیفیت مناسب دست یابند.
	\end{itemize}
}

\subsectionaddtolist{ج}{

\subsubsection*{اجزای کلیدی معماری}
DSLAM دستگاهی است که سیگنال‌های دیجیتال دریافتی از کاربران DSL را جمع‌آوری و به یک اتصال پرسرعت به سمت شبکه مرکزی (core-network) تبدیل می‌کند.
  معمولاً در مرکز مخابراتی یا در کابین‌های خیابانی نزدیک به کاربران نصب می‌شود.
  وظیفه آن  تبدیل خطوط تلفن آنالوگ به سیگنال‌های دیجیتال و مدیریت ترافیک چندین کاربر DSL است.



CMTS در سمت ارائه‌دهنده خدمات قرار دارد و داده‌ها را بین اینترنت و مودم‌های کابلی در منازل کاربران رد و بدل می‌کند.
 در مرکز عملیات شبکه (Headend) نصب میشود. وظیفه آن  مدیریت اتصالات کاربران، اختصاص پهنای باند، کنترل ترافیک اینترنت کابلی است.
 
 OLT نقطه مرکزی شبکه فیبر نوری است که ارتباط بین اینترنت و چندین ONU/ONT (دستگاه‌های کاربران) را مدیریت می‌کند.
  در مرکز سرویس‌دهی اپراتور نصب میشود.
  وظیفه آن  تبدیل سیگنال‌های نوری به الکتریکی و بالعکس، و ارسال/دریافت داده‌ها از کاربران از طریق شبکه GPON یا EPON است.

  
  \subsubsection*{اثرات خرابی}
  
  اگر DSLAM خراب شود:
  
  \begin{itemize}
  	\item همه کاربران DSL متصل به آن دستگاه اتصال اینترنت خود را از دست می‌دهند.
  	\item ممکن است خطوط تلفن ثابت نیز دچار اختلال شوند (در صورت استفاده‌ی مشترک از تجهیزات).
  	\item اختلال فقط روی محدوده‌ای مشخص از کاربران تأثیر می‌گذارد ( وابسته به پوشش آن DSLAM ).
  \end{itemize}
 
 اگر CMTS یا OLT خراب شود:
 
 \begin{itemize}
 	\item CMTS : قطع ارتباط برای همه کاربران کابل متصل به آن بخش.
 	\item OLT : از کار افتادن کل زیرشبکه فیبر که به آن OLT وصل است (صدها کاربر ممکن است تحت تأثیر قرار بگیرند).
 \end{itemize}


\subsubsection*{جلوگیری از خرابی}

\begin{enumerate}
	\item طراحی با Redundancy :
	\begin{itemize}
		\item استفاده از DSLAM، CMTS یا OLT اضافی (Backup) برای پشتیبانی در صورت خرابی دستگاه اصلی.
		\item ایجاد مسیرهای متنوع برای اتصال کاربران به بیش از یک تجهیز اصلی.
	\end{itemize}
	\item معماری توزیع‌شده
	\begin{itemize}
		\item استفاده از چند DSLAM یا OLT با پوشش‌های کوچک‌تر به‌جای یک دستگاه مرکزی بزرگ.
		\item کاهش تعداد کاربرانی که در صورت خرابی یک تجهیز دچار اختلال می‌شوند.
	\end{itemize}
	\item دسترسی چندگانه:
	\begin{itemize}
		\item در برخی مناطق، ارائه دسترسی ترکیبی ( مثلاً هم Wi-Fi و هم DSL ) برای سوئیچ خودکار به مسیر پشتیبان در صورت قطعی.
	\end{itemize}
\end{enumerate}


}


\subsectionaddtolist{د}{

\subsubsection*{مناطق پرتراکم – شبکه سیمی (FTTH/LAN) + Wi-Fi}

برای ساختمان‌ها از FFTB یا Ethernet استفاده شود و در فضاهای داخلی پرتردد از Wi-Fi .

$\\$
توجیه:

\begin{itemize}
	\item فیبر/سیمی: ارائه‌ی سرعت بسیار بالا برای استریم، آپلود فایل‌های حجیم، و کلاس‌های آنلاین.
	\item Wi-Fi : اتصال موبایل و لپ‌تاپ دانشجویان با سهولت و پوشش بالا در فضاهای عمومی مثل کتابخانه یا سلف.
\end{itemize}


\subsubsection*{ مناطق کم‌تراکم – اتصال بی‌سیم ( FWA / Outdoor-Wi-Fi )}

\begin{itemize}
	\item سیم‌کشی به این مناطق (مثل زمین‌های ورزشی یا مراکز تحقیقاتی دورافتاده) پرهزینه و غیرعملی است.
	\item با نصب آنتن‌های بی‌سیم نقطه به نقطه یا سلولی، دسترسی به اینترنت با تأخیر کم و هزینه پایین حاصل می‌شود.
\end{itemize}

\subsubsection*{ارتباط بین تجهیزات اصلی}
\begin{itemize}
	\item فیبر نوری  Backbone برای اتصال همه بخش‌ها به دیتاسنتر یا شبکه اصلی دانشگاه.
	\item سوئیچ‌های توزیع و هاب‌های محلی برای مدیریت ترافیک.
\end{itemize}

}

