\subsectionaddtolist{الف}


اینترنت به دلیل مقیاس عظیم و نیاز به استقلال مدیریتی، به یک ساختار سلسله‌مراتبی از سیستم‌های خودمختار تقسیم شده است. این تفکیک، ایجاب می‌کند که از دو نوع پروتکل مسیریابی با اهداف کاملاً متفاوت استفاده شود:

\begin{itemize}

\item{ مقیاس‌پذیری و کنترل سربار:}

یک پروتکل واحد برای کل اینترنت غیرممکن است. اگر تمام روترهای جهان در یک دامنه مسیریابی قرار داشتند، حجم اطلاعات مسیریابی (تغییرات لینک، آپدیت‌ها) به حدی زیاد می‌شد که هیچ روتری توان پردازش آن را نداشت و شبکه به دلیل سربار ارتباطی فلج می‌شد. تقسیم شبکه به {AS} ها باعث می‌شود که پروتکل‌های {درون ناحیه‌ای IGP} تنها اطلاعات مربوط به شبکه داخلی خود را مدیریت کنند و از انتشار جزئیات غیرضروری به کل اینترنت جلوگیری نمایند.

\item{ استقلال مدیریتی و اعمال سیاست:}

هر {AS} توسط یک سازمان مستقل مدیریت می‌شود و می‌خواهد کنترل کاملی بر ترافیک ورودی و خروجی خود داشته باشد. پروتکل‌های بین ناحیه‌ای، برای اعمال سیاست‌های پیچیده طراحی شده‌اند. این سیاست‌ها می‌توانند بر اساس توافقات تجاری، مسائل امنیتی یا هزینه‌ها باشند. در مقابل، پروتکل‌های درون ناحیه‌ای، هیچ ابزاری برای درک یا اعمال چنین سیاست‌هایی ندارند.

\item{ اهداف متفاوت بهینه‌سازی:}

هدف در مسیریابی {درون ناحیه‌ای}، معمولاً یک هدف فنی است: یافتن سریع‌ترین یا کوتاه‌ترین مسیر در داخل یک شبکه. اما در سطح {بین ناحیه‌ای}، بهترین مسیر لزوماً سریع‌ترین مسیر نیست. بهترین مسیر ممکن است مسیری باشد که هزینه کمتری دارد، از طریق یک شریک تجاری معتبر عبور می‌کند، یا از عبور از شبکه یک رقیب اجتناب می‌کند. هدف در اینجا بهینه‌سازی بر اساس {سیاست} است، نه عملکرد فنی.
\end{itemize}


\subsectionaddtolist{ب}


خیر، یک روتر {BGP} همیشه مسیری با کوتاه‌ترین طول \lr{AS-PATH} را انتخاب {نمی‌کند}.

هدف اصلی {BGP} یافتن سریع‌ترین یا کوتاه‌ترین مسیر نیست، بلکه {اعمال سیاست‌های} مدیریتی و تجاری است. برای این منظور، {BGP} از یک الگوریتم تصمیم‌گیری مرحله به مرحله استفاده می‌کند که در آن چندین ویژگی به ترتیب اولویت بررسی می‌شوند. طول مسیر \lr{AS-PATH} تنها یکی از این ویژگی‌هاست و در مراحل اولیه قرار ندارد.

فرآیند کلی انتخاب مسیر به صورت زیر است:
\begin{enumerate}
	\item
	 {بررسی ویژگی‌های با اولویت بالا:} 
	 روتر ابتدا ویژگی‌هایی مانند {\lr{LOCAL\_PREF}} را بررسی می‌کند. یک مسیر با \lr{LOCAL\_PREF} بالاتر، {صرف‌نظر از طول \lr{AS-PATH} آن}، همیشه برنده خواهد بود. این ویژگی ابزار اصلی برای اعمال سیاست در یک سیستم خودمختار است.
	
	\item {بررسی طول \lr{AS-PATH} :} 
	تنها در صورتی که تمام ویژگی‌های با اولویت بالاتر برای دو یا چند مسیر یکسان باشند، روتر به سراغ بررسی طول \lr{AS-PATH} می‌رود و در این مرحله، مسیر با طول کوتاه‌تر را انتخاب می‌کند.
	
	\item {سایر معیارها:}
	 اگر طول مسیرها نیز یکسان باشد، معیارهای دیگری مانند نوع مسیر و... برای شکستن تساوی به کار می‌روند.
\end{enumerate}


اطمینان از بدون حلقه بودن مسیرها در {BGP} به صورت مستقل و پیش از فرآیند انتخاب مسیر انجام می‌شود. یک روتر هر آپدیت مسیری را که دریافت می‌کند، بررسی کرده و اگر شماره \lr{AS} خود را در لیست \lr{AS-PATH} آن مسیر ببیند، آن را به طور کامل نادیده می‌گیرد تا از ایجاد حلقه جلوگیری کند. بنابراین، تمام مسیرهایی که وارد فرآیند انتخاب مسیر می‌شوند، از قبل تضمین شده است که بدون حلقه هستند.

{نتیجه‌گیری:} مسیر انتخابی توسط {BGP} همواره بدون حلقه است، اما به دلیل اولویت بالاتر سیاست‌ها، این مسیر لزوماً کوتاه‌ترین مسیر از نظر تعداد \lr{AS}ها نیست.




\subsectionaddtolist{ج}



هر روتر بر اساس موقعیت خود در شبکه (داخلی یا مرزی) و نوع پیشوند (داخلی یا خارجی)، از پروتکل متفاوتی برای یادگیری مسیر استفاده می‌کند.

\begin{itemize}
	
	\item روتر \lr{3a}:

این روتر با eBGP از روتر
\lr{4c}
یاد می‌گیرد.


	\item روتر \lr{1c}:
	
	این روتر با eBGP از روتر
	\lr{4a}
	یاد می‌گیرد.
	
		\item روتر \lr{3c}:
	
	این روتر با iBGP از روتر
	\lr{3a}
	یاد می‌گیرد.
	
\end{itemize}


\subsectionaddtolist{د}


رابط روی \textbf{\lr{I2}} تنظیم خواهد شد.


این انتخاب صورت می‌گیرد زیرا الگوریتم \lr{BGP} در \lr{AS1}، پس از بررسی سیاست‌های محلی، مسیر از طریق \lr{AS3} را به دلیل داشتن طول مسیر \lr{AS-PATH} کوتاه‌تر یا دیگر ویژگی‌های برتر، بهینه تشخیص داده است.


تصمیم‌گیری بین دو مسیر دریافتی برای پیشوند \lr{x} (یکی از سمت \lr{AS2} و دیگری از سمت \lr{AS3})، کاملاً به {الگوریتم انتخاب مسیر \lr{BGP}} در \lr{AS1} بستگی دارد. این فرآیند سلسله‌مراتبی است:

\begin{enumerate}
	\item {اولویت با سیاست است:} ابتدا، \lr{BGP} ویژگی‌های سیاستی مانند {\lr{LOCAL\_PREF}} را بررسی می‌کند. مدیر شبکه \lr{AS1} می‌تواند با تنظیم \lr{LOCAL\_PREF} بالاتر برای مسیری که از \lr{AS3} می‌آید، تمام روترهای داخلی از جمله \lr{d1} را مجبور کند که رابط \lr{I2} را انتخاب نمایند، حتی اگر آن مسیر طولانی‌تر باشد.
	
	\item {طول مسیر به عنوان معیار دوم:} تنها در صورتی که مقادیر \lr{LOCAL\_PREF} برای هر دو مسیر یکسان باشند، \lr{BGP} به سراغ معیار بعدی، یعنی طول {\lr{AS-PATH}}، می‌رود. در این حالت، هر مسیری که تعداد \lr{AS}های کمتری در مسیر خود داشته باشد، به عنوان مسیر بهینه انتخاب خواهد شد.
\end{enumerate}

بنابراین، انتخاب نهایی به تنظیمات و سیاست‌های خاصی که در \lr{AS1} پیکربندی شده است، بستگی دارد و انتخاب \lr{I2} به معنای برتری آن مسیر بر اساس این معیارهاست.
