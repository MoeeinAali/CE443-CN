\subsectionaddtolist{الف}

\[
\begin{array}{|c|c|c|c|c|c|}
	\hline
	\textbf{} & Dc(e) & Dc(d) & Dc(c) & Dc(b) & Dc(a) \\
	\hline
	t=1 & \infty & 0 & \infty & \infty & \infty \\
	\hline
	t=2 & \infty & 2 & 0 & 6 & \infty \\
	\hline
	t=3 & 5 & 2 & 0 & 5 & 7 \\
	\hline
	t=4 & 5 & 2 & 0 & 5 & 6 \\
	\hline
\end{array}
\]

\subsectionaddtolist{ب}

در پروتکل‌های مسیریابی بردار فاصله، مشکل شمارش تا بی‌نهایت تنها زمانی رخ می‌دهد که هزینه یک لینک {افزایش} یابد یا لینکی به طور کامل {قطع} شود. در شرایط دیگر، این مشکل بروز نمی‌کند.

\subsubsection*{حالت اول: کاهش هزینه یک لینک}

گره‌های مجاور به سرعت هزینه جدید و کمتر را از طریق تبادل اطلاعات دریافت می‌کنند. این خبر خوب به سرعت در شبکه منتشر می‌شود. هر گره، با دریافت آپدیت جدید، جدول مسیریابی خود را به‌روزرسانی کرده و مسیر بهینه‌تر را انتخاب می‌کند. این فرآیند منجر به همگرایی سریع شبکه به یک حالت پایدار جدید می‌شود و لوپ بی‌نهایت ایجاد نمی‌شود.
	

\subsubsection*{حالت دوم: اتصال دو گره جدید}

برقراری یک لینک جدید، مسیرهای جدید و کوتاه‌تری را به شبکه معرفی می‌کند. این خبر خوب توسط گره‌های دو سر لینک به سرعت در شبکه منتشر می‌شود. سایر گره‌ها با دریافت این اطلاعات، جداول مسیریابی خود را به‌روزرسانی کرده و مسیرهای بهینه‌تر جدید را کشف می‌کنند. این فرآیند نیز منجر به همگرایی سریع شده و از ایجاد لوپ بی‌نهایت جلوگیری می‌کند.



\subsectionaddtolist{ج}



\subsubsection*{الگوریتم‌های متمرکز (\lr{Centralized})}
\begin{itemize}
	\item {ویژگی‌ها:} در این مدل، تمام اطلاعات توپولوژی و وضعیت لینک‌ها به یک گره مرکزی (کنترل‌کننده مسیر) ارسال می‌شود. این گره دیدی کامل و سراسری از شبکه دارد و تمام مسیرها را از این دیدگاه جامع محاسبه می‌کند.
	
	\item {مزایا و معایب:} مزیت اصلی، توانایی محاسبه {مسیرهای کاملاً بهینه} است. اما این ساختار یک  \lr{Single Point of Failure} ایجاد می‌کند؛ اگر کنترلر از کار بیفتد، کل فرآیند مسیریابی مختل می‌شود. همچنین با بزرگ شدن شبکه، با {چالش‌های مقیاس‌پذیری} روبرو می‌شود.
	
	\item {مثال:} معماری {شبکه‌های نرم‌افزارمحور (\lr{SDN})} که در آن یک کنترلر مرکزی مسئولیت تصمیم‌گیری و مدیریت جریان ترافیک را بر عهده دارد.
\end{itemize}

\subsubsection*{الگوریتم‌های توزیع‌شده (\lr{Distributed})}
\begin{itemize}
	\item {ویژگی‌ها:} منطق مسیریابی بین تمام روترها تقسیم می‌شود. هر گره مسیرهای خود را بر اساس {اطلاعات محلی} و داده‌هایی که به صورت {تکرارشونده از همسایگان مستقیم} خود دریافت می‌کند، محاسبه می‌نماید. هیچ گرهی نقشه کامل شبکه را در اختیار ندارد.
	
	\item {مزایا و معایب:} این رویکرد به دلیل نداشتن نقطه شکست واحد، بسیار \textbf{مقاوم و مقیاس‌پذیر} است. با این حال، چون تصمیمات با اطلاعات ناقص گرفته می‌شود، فرآیند {همگرایی} پس از تغییرات شبکه کندتر است و ممکن است به مشکلات موقتی مانند {حلقه‌های مسیریابی} منجر شود.
	
	\item {مثال:} پروتکل‌های خانواده {بردار فاصله (\lr{Distance-Vector})} مانند {\lr{RIP}} و پروتکل‌های \lr{Link-State} مانند {\lr{OSPF}} که در آن هر روتر به طور مستقل الگوریتم دایجسترا را اجرا می‌کند.
\end{itemize}
