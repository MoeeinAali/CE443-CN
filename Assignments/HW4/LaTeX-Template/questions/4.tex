\subsectionaddtolist{الف}


در پروتکل OpenFlow ، هر {ورودی جریان (\lr{Flow Entry})} مجموعه‌ای از پارامترها و قوانین است که تعیین می‌کند چگونه یک سوئیچ با بسته‌های دریافتی رفتار کند. این ورودی‌ها در جدول‌های جریان (\lr{Flow Tables}) نگهداری می‌شوند و به عنوان هسته تصمیم‌گیری سوئیچ‌های OpenFlow عمل می‌کنند. ساختار یک ورودی جریان شامل سه بخش اصلی زیر است:

\begin{itemize}
	\item {\textbf{Match-Fields}}: این فیلدها مشخص می‌کنند که چه ویژگی‌هایی از بسته باید بررسی شود تا مشخص شود که آیا بسته با این جریان مطابقت دارد یا خیر. این ویژگی‌ها می‌توانند شامل آدرس MAC، آدرس IP، شماره پورت، پروتکل لایه انتقال و یا شماره پورت ورودی سوئیچ باشند. این بخش نقش کلیدی در شناسایی جریان‌ها و دسته‌بندی ترافیک ایفا می‌کند.
	
	\item \textbf{Counters}: این بخش شامل اطلاعات آماری مانند تعداد بسته‌ها یا حجم داده‌هایی است که با این جریان تطابق داشته‌اند. شمارنده‌ها برای پایش عملکرد شبکه و تحلیل ترافیک به کار می‌روند و می‌توانند به کنترلر یا مدیر شبکه در تصمیم‌گیری برای بهینه‌سازی سیاست‌ها کمک کنند.
	
	\item \textbf{Instructions}: این بخش مشخص می‌سازد که در صورت مطابقت بسته با این جریان، چه اقداماتی باید انجام گیرد. دستورالعمل‌ها می‌توانند شامل ارسال بسته به خروجی خاص، هدایت به جدول جریان دیگر، تغییر برخی از فیلدهای بسته، یا حتی ارسال آن به کنترلر باشند. این بخش، مسئول اجرای سیاست‌های شبکه بر اساس تحلیل و تصمیم‌گیری قبلی است.
\end{itemize}

\subsectionaddtolist{ب}



در معماری شبکه SDN ، کنترل‌کننده مرکزی دارای ساختاری چندلایه است که هر لایه وظایف خاصی را برای مدیریت هوشمندانه شبکه بر عهده دارد. سه لایه‌ی مهم آن عبارتند از:

\begin{enumerate}
	\item \textbf{لایه ارتباطات:} \\
	این لایه وظیفه برقراری ارتباط مستقیم با تجهیزات صفحه داده، مانند سوئیچ‌ها و روترها را دارد. از طریق پروتکل‌هایی نظیر OpenFlow ، این لایه اطلاعات کنترلی (مانند قوانین ارسال بسته‌ها) را به دستگاه‌های زیرساخت ارسال کرده و اطلاعات وضعیت (مانند جریان‌های ترافیکی و هشدارها) را از آن‌ها دریافت می‌کند. به‌طور خلاصه، این لایه پل ارتباطی میان کنترل‌کننده و شبکه فیزیکی است.
	
	\item \textbf{لایه مدیریت وضعیت سراسری شبکه:} \\
	این بخش به جمع‌آوری داده‌های شبکه، مانند توپولوژی، وضعیت لینک‌ها، جدول‌های جریان و آمار ترافیکی می‌پردازد. هدف آن ایجاد یک تصویر جامع و به‌روز از وضعیت کل شبکه است. اطلاعاتی که در این لایه ذخیره و پردازش می‌شوند، پایه تصمیم‌گیری برای برنامه‌های کنترلی در لایه‌های بالاتر را فراهم می‌کنند.
	
	\item \textbf{لایه برنامه‌های کنترلی شبکه:} \\
	در این لایه، برنامه‌های کاربردی شبکه اجرا می‌شوند که وظیفه آن‌ها تحلیل وضعیت شبکه و اعمال سیاست‌های مدیریتی است. این برنامه‌ها با استفاده از API هایی که کنترل‌کننده در اختیارشان قرار می‌دهد، می‌توانند عملکرد شبکه را در زمینه‌هایی مانند مسیریابی پویا، مدیریت پهنای باند، امنیت یا کیفیت خدمات بهینه‌سازی کنند. در واقع، این لایه هوش شبکه را تشکیل می‌دهد.
\end{enumerate}

این سه لایه به صورت یکپارچه عمل کرده و به کنترل‌کننده SDN اجازه می‌دهند تا دیدی متمرکز و قابل برنامه‌ریزی از کل شبکه داشته باشد.

 \pagebreak
 
 

\subsectionaddtolist{ج}



در معماری SDN ، کنترلر مرکزی قادر است با استفاده از پروتکل OpenFlow سیاست‌های امنیتی مورد نظر را روی شبکه پیاده‌سازی کند. این کار از طریق تعامل مستقیم با سوئیچ‌های صفحه داده و استفاده از پیام‌های کنترلی مختلف صورت می‌گیرد. فرآیند کلی اجرای یک سیاست امنیتی برای جریان‌های ناشناس جدید به شرح زیر است:

\begin{enumerate}
	\item \textbf{دریافت اولین بسته توسط سوئیچ:} \\
	زمانی‌که یک بسته‌ی جدید که مربوط به جریانی ناشناخته است وارد سوئیچ می‌شود، سوئیچ نمی‌تواند آن را با هیچ‌کدام از ورودی‌های جریان خود تطبیق دهد. در نتیجه، بر اساس تنظیمات پیش‌فرض، سوئیچ این بسته را در قالب یک پیام {Packet-In} به کنترلر ارسال می‌کند. این پیام حاوی اطلاعات کلیدی مانند سربرگ بسته، شماره پورت ورودی و مشخصات جریان است.
	
	\item \textbf{تحلیل و تصمیم‌گیری کنترلر:} \\
	کنترلر پس از دریافت پیام Packet-In ، با توجه به سیاست امنیتی تعریف‌شده (برای مثال بررسی لیست‌های مجاز یا غیرمجاز، قوانین فایروال، یا تحلیل ترافیک)، اقدام به تحلیل اطلاعات موجود در بسته می‌کند. سپس تصمیم می‌گیرد که آیا جریان جدید باید مجاز شناخته شود یا مسدود گردد.
	
	\item \textbf{ارسال پیام FlowMod :} \\
	اگر کنترلر تصمیم به پذیرش جریان بگیرد، یک پیام {FlowMod} به سوئیچ ارسال می‌کند. این پیام شامل ورودی جدیدی برای جدول جریان سوئیچ است که مشخص می‌کند بسته‌های مشابه در آینده چگونه پردازش شوند (مثلاً به کدام پورت ارسال یا حذف شوند).
	
	\item \textbf{ارسال Packet-Out :} \\
	برای جلوگیری از تأخیر در پردازش اولین بسته، کنترلر ممکن است علاوه بر FlowMod ، یک پیام {Packet-Out} نیز ارسال کند تا همان بسته‌ای که ابتدا دریافت شده بود، فوراً به مسیر مقصد هدایت شود. این کار باعث می‌شود جریان از همان ابتدا بدون وقفه پاسخ بگیرد.
	
	\item \textbf{اعمال سیاست توسط سوئیچ:} \\
	پس از دریافت FlowMod ، سوئیچ ورودی جدید را در جدول خود ذخیره می‌کند. از این پس، بسته‌های بعدی که متعلق به همان جریان باشند، بدون نیاز به تماس با کنترلر، مستقیماً و مطابق با سیاست اعمال‌شده هدایت می‌شوند.
	
\end{enumerate}

در مجموع، این تعامل مبتنی بر پیام‌های OpenFlow به کنترلر اجازه می‌دهد تا کنترل دقیق و متمرکزی بر امنیت و مدیریت جریان‌های داده در شبکه داشته باشد، بدون آنکه به صورت مداوم درگیر پردازش هر بسته شود.

\subsectionaddtolist{د}


در معماری کنترلر {OpenDaylight} ، لایه‌ای با نام {لایه انتزاع سرویس} وجود دارد که نقش کلیدی در ساده‌سازی تعامل میان برنامه‌های کاربردی و پروتکل‌های زیرساختی ایفا می‌کند.

هدف اصلی این لایه، ایجاد یک واسط انتزاعی و یکنواخت برای برنامه‌های کنترلی شبکه است؛ به‌ طوری‌ که توسعه‌دهندگان بتوانند بدون درگیری با پیچیدگی‌های جزئیات پیاده‌سازی پروتکل‌های مختلف نظیر OpenFlow ، NETCONF یا BGP به توسعه منطق شبکه بپردازند.

این لایه مانند یک مترجم بین دو دنیا عمل می‌کند: از یک سو، درخواست‌ها و دستورات صادرشده از سوی برنامه‌های کاربردی را دریافت می‌کند و از سوی دیگر، آن‌ها را به شکل سازگار با پروتکل‌های سطح پایین تبدیل کرده و به تجهیزات شبکه ارسال می‌کند. 

این ساختار ماژولار نه تنها توسعه‌ی برنامه‌ها را تسهیل می‌کند، بلکه قابلیت {قابل حمل بودن} کدها را افزایش داده و امکان استفاده از فناوری‌های مختلف در لایه‌های پایین‌تر را بدون نیاز به بازنویسی برنامه‌های کنترلی فراهم می‌آورد.


\subsectionaddtolist{ه}


در شبکه‌های مبتنی بر {SDN} ، کنترلر مرکزی نقش کلیدی در مدیریت پویای وضعیت شبکه دارد. در صورت بروز خرابی در یکی از لینک‌ها، زنجیره‌ای از تعاملات میان سوئیچ‌ها، کنترلر و برنامه‌های کنترلی رخ می‌دهد تا مسیرهای جدید تعیین شده و شبکه مجدداً به وضعیت پایدار بازگردد. این فرآیند شامل مراحل زیر است:

\begin{enumerate}
	\item \textbf{شناسایی خرابی توسط سوئیچ:} \\
	هنگامی که یکی از لینک‌های متصل به یک سوئیچ دچار مشکل شود، سوئیچ با استفاده از مکانیزم‌هایی نظیر {LLDP} یا مانیتورینگ لایه فیزیکی، خرابی را تشخیص می‌دهد.
	
	\item \textbf{اعلان رویداد به کنترلر:} \\
	سوئیچ با ارسال یک پیام به کنترلر، وقوع خرابی را گزارش می‌دهد. این پیام شامل اطلاعاتی در مورد پورت غیرفعال‌شده و وضعیت آن است.
	
	\item \textbf{به‌روزرسانی وضعیت توپولوژی:} \\
	کنترلر، در لایه مدیریت وضعیت شبکه، وضعیت توپولوژی را به‌روزرسانی می‌کند و لینک معیوب را از نقشه توپولوژی خود حذف می‌نماید. در صورت وجود داده‌های آماری، ممکن است این مرحله شامل بررسی نرخ از دست رفتن بسته نیز باشد.
	
	\item \textbf{اطلاع‌رسانی به برنامه‌های کنترلی:} \\
	تغییر توپولوژی به اطلاع برنامه‌هایی مانند \textbf{مسیریابی دینامیک}، {تعادل بار} یا کنترل کیفیت خدمات می‌رسد. این برنامه‌ها از وضعیت جدید شبکه مطلع شده و وارد عمل می‌شوند.
	
	\item \textbf{محاسبه مسیرهای جایگزین:} \\
	برنامه‌های کنترلی با استفاده از توپولوژی به‌روز شده، مسیرهای جدیدی برای جریان‌های فعلی که از لینک آسیب‌دیده عبور می‌کردند محاسبه می‌کنند. این مسیرها ممکن است بر مبنای معیارهایی چون {کمترین تأخیر} یا {کمترین بار} تعیین شوند.
	
	\item \textbf{اعمال تغییرات در سوئیچ‌ها:} \\
	کنترلر با استفاده از پیام‌های {FlowMod} ، ورودی‌های جریان جدید را به سوئیچ‌های مربوطه ارسال می‌کند تا بسته‌ها از مسیرهای تازه هدایت شوند.
	
	\item \textbf{حذف قوانین قبلی:} \\
	در کنار ارسال قوانین جدید، کنترلر می‌تواند پیام‌هایی برای حذف قوانین قدیمی مرتبط با مسیر قبلی نیز ارسال کند تا از بار اضافی بر روی سوئیچ‌ها جلوگیری شود.
	
	\item \textbf{بازگشت به وضعیت پایدار:} \\
	پس از به‌روزرسانی ورودی‌های جریان در سوئیچ‌ها، مسیرهای جدید فعال شده و ترافیک مجدداً به صورت روان منتقل می‌شود. بدین ترتیب، شبکه به وضعیت پایدار جدیدی بازمی‌گردد.
\end{enumerate}

\textbf{نقش اجزای مختلف در این فرآیند:}

\begin{itemize}
	\item \textbf{سوئیچ‌ها:} وظیفه تشخیص خطای لینک و ارسال گزارش به کنترلر، و اجرای قوانین جدید از سوی کنترلر را برعهده دارند.
	
	\item \textbf{کنترلر (لایه ارتباطات و وضعیت):} دریافت پیام‌های رویداد، به‌روزرسانی توپولوژی شبکه، و ارسال FlowMod به سوئیچ‌ها را مدیریت می‌کند.
	
	\item \textbf{برنامه‌های کنترلی شبکه:} بر مبنای توپولوژی جدید، تصمیم‌های مدیریتی (مانند بازتنظیم مسیرها) اتخاذ می‌کنند و به کنترلر فرمان می‌دهند.
\end{itemize}

